\documentclass[12pt]{article}
\usepackage{geometry}
\usepackage{amsmath, amssymb}
\usepackage{graphicx}
\usepackage{hyperref}
\geometry{margin=1in}

\title{Detailed Summary and Analysis of: \\ \textbf{Online Fault Monitoring Based on Deep Neural Network and Sliding Window Technique}}
\author{}
\date{}

\begin{document}

\maketitle

\section*{Objective}
This study presents an \textbf{online fault monitoring system} for Nuclear Power Plants (NPPs) using \textbf{Deep Convolutional Neural Networks (DCNN)} integrated with a \textbf{Sliding Window (SW)} technique. The system aims to enhance diagnostic accuracy, robustness, and reliability across various plant states and fault severities.

\section*{Key Contributions}
\begin{itemize}
    \item Proposed a 2D-DCNN based online monitoring system suitable for trend-based time-series data from NPPs.
    \item Integrated a Sliding Window technique to enable adaptive, real-time feature extraction.
    \item Demonstrated superiority over SVM, RBFNN through comparative analysis using realistic data sets.
    \item Employed LSTM for fault severity estimation and incorporated a simulation-based verification mechanism.
\end{itemize}

\section*{Methodology Overview}

\subsection*{1. Reactor Model: IP-200}
\begin{itemize}
    \item A small modular IPWR developed by Harbin Engineering University.
    \item Thermal-hydraulic simulation performed using RELAP5.
    \item Validation through steady-state parameter accuracy.
\end{itemize}

\subsection*{2. Deep Convolutional Neural Network (DCNN)}
\begin{itemize}
    \item Multi-layer structure with convolution, batch normalization, ReLU, pooling, and softmax layers.
    \item Designed for 2D time-series input: Input size = 43x100, Output size = 6 classes.
    \item Training: SGDM with early stopping; epochs = 50; learning rate = 0.0001.
\end{itemize}

\subsection*{3. Sliding Window Technique}
\begin{itemize}
    \item Dynamically updates network parameters with the most recent plant data.
    \item Supports adaptability under varying power levels and transient states.
\end{itemize}

\section*{Performance Comparison}

\subsection*{DCNN vs SVM}
\begin{itemize}
    \item In-sample: Both models perform equally well.
    \item Out-sample: SVM misclassifies 257/600 cases due to sensitivity to minor noise; DCNN maintains accuracy.
\end{itemize}

\subsection*{LSTM vs RBFNN for Severity Estimation}
\begin{itemize}
    \item In-sample MSE: LSTM = 1.81e-4, RBFNN = 2.11e-4
    \item Out-sample MSE: LSTM = 7.58e-4, RBFNN = 0.11
\end{itemize}

\section*{Proposed Fault Monitoring Architecture}
\begin{enumerate}
    \item \textbf{Offline Training}: Data sorted and normalized, CNN trained.
    \item \textbf{Online Monitoring}: Incoming data preprocessed; CNN detects faults.
    \item \textbf{Severity Analysis}: Corresponding LSTM model estimates fault severity.
    \item \textbf{Verification}: Simulator compares real plant data with predicted outcomes to compute MSE.
\end{enumerate}

\section*{Experimental Scenarios}
\begin{itemize}
    \item 5 fault classes simulated at different power levels.
    \item Test cases: e.g., SGTR at 68\% power with 14\% severity; PORV stuck at 33\% power with 78\% valve position.
    \item Achieved MSE: 0.0027 (SGTR), 0.000010384 (PORV).
\end{itemize}

\section*{Conclusion}
\begin{itemize}
    \item The proposed DCNN + SW + LSTM framework demonstrates robust performance under dynamic and noisy conditions.
    \item Adaptive feature extraction ensures real-time diagnostic capability.
    \item Validation via simulation reinforces confidence in fault detection and severity estimation.
    \item Outperforms traditional ML models (SVM, RBFNN) in both accuracy and generalizability.
\end{itemize}

Future work will aim to integrate this system into live NPPs for real-world validation.

\end{document}
