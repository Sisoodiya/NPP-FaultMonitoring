\documentclass[12pt]{article}
\usepackage{geometry}
\usepackage{amsmath, amssymb}
\usepackage{graphicx}
\usepackage{hyperref}
\geometry{margin=1in}

\title{Detailed Summary and Analysis of: \\ \textbf{Intelligent Fault Monitoring and Reliability Analysis in Safety--Critical Systems of Nuclear Power Plants Using SIAO-CNN-ORNN}}
\author{}
\date{}

\begin{document}

\maketitle

\section*{Objective}
This paper proposes a novel deep learning-based model, \textbf{SIAO-CNN-ORNN}, to detect faults and predict the reliability of Instrumentation and Control (I\&C) systems in Nuclear Power Plants (NPPs). The model integrates:
\begin{itemize}
    \item Convolutional Neural Networks (CNN) for feature learning,
    \item Optimized Recurrent Neural Networks (ORNN) for temporal analysis,
    \item Self-Improved Aquila Optimizer (SIAO) for parameter optimization.
\end{itemize}

\section*{Key Contributions}
\begin{itemize}
    \item Introduced a hybrid deep learning model (CNN + ORNN) for fault monitoring.
    \item Proposed \textbf{Weighted Kurtosis and Skewness (WKS)} for enhanced feature extraction.
    \item Implemented the Self-Improved Aquila Optimizer (SIAO) to improve RNN weight updates.
    \item Demonstrated system reliability estimation based on failure rate, MTTF, and dynamic analysis.
\end{itemize}

\section*{Methodology Overview}

\subsection*{1. Data Acquisition}
Simulation data from the IP-200 modular reactor using RELAP5 covering:
\begin{itemize}
    \item Steady state
    \item Pressurizer PORV stuck open
    \item Steam generator tube rupture
    \item Feedwater line break
    \item Reactor coolant pump (RCP) failure
\end{itemize}

\subsection*{2. Data Pre-processing}
Cleaning, removing redundancy, handling missing/null values.

\subsection*{3. Feature Extraction}
Statistical operations including:
\begin{itemize}
    \item Mean, Median, SD, Variance
    \item Skewness, Kurtosis
    \item \textbf{Weighted Kurtosis and Skewness (WKS)} optimized using Aquila Optimizer
\end{itemize}

\subsection*{4. Hybrid Model: CNN + ORNN}
\begin{itemize}
    \item \textbf{CNN:} Automated feature extraction from input data.
    \item \textbf{RNN:} Processes sequential temporal data.
    \item \textbf{SIAO:} Optimizes RNN weights via chaotic-enhanced Aquila optimization.
\end{itemize}

\section*{Self-Improved Aquila Optimizer (SIAO)}
\begin{itemize}
    \item Simulates four hunting behaviors: expanded/narrowed exploration and exploitation.
    \item Uses chaotic maps to improve exploration and avoid local minima.
    \item Objective function: Root Mean Square Error (RMSE)
\end{itemize}

\section*{Reliability Estimation}
\textbf{Key formulas:}
\begin{align*}
\text{Failure Rate (}\lambda\text{)} &= \frac{\text{Number of Failures}}{\text{Operating Time}} \\
\text{Reliability} &= 1 - \lambda \\
\text{MTTF} &= \frac{1}{\lambda} \\
\text{Reliability over Time} &= e^{-T/\text{MTTF}}
\end{align*}

\section*{Results and Evaluation}
\textbf{Performance Comparison:}
\begin{itemize}
    \item Accuracy: \textbf{98.74\%} (SIAO-CNN-ORNN)
    \item Sensitivity: 0.9538\quad Specificity: 0.9885
    \item F1-Measure: 0.9538\quad MCC: 0.9423
    \item RMSE: 0.00891\quad MAE: 0.0312
\end{itemize}

\textbf{Comparison with other techniques:}
\begin{itemize}
    \item CNN-RNN: Accuracy = 96.23\%
    \item CNN: 93.12\% \quad Bi-LSTM: 94.05\% \quad LSTM: 91.18\%
\end{itemize}

\section*{Conclusion}
The proposed SIAO-CNN-ORNN model successfully integrates deep learning and metaheuristic optimization to:
\begin{itemize}
    \item Enhance fault detection in safety-critical NPP systems
    \item Dynamically predict and monitor reliability metrics
    \item Outperform existing models in accuracy and robustness
\end{itemize}
This makes it a powerful tool for future deployment in nuclear I\&C systems to ensure safe and efficient operation.

\end{document}
